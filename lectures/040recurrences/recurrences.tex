\documentclass{article}
\usepackage{fullpage}

\usepackage{amsmath,cancel}
\usepackage{graphicx}
\usepackage{color}
\usepackage{alltt}
\newcommand{\red}[1]{{\color{red}#1}}
\newcommand{\cyan}[1]{{\color{cyan}#1}}
\newcommand{\blue}[1]{{\color{blue}#1}}
\newcommand{\magenta}[1]{{\color{magenta}#1}}
\newcommand{\yellow}[1]{{\color{yellow}#1}}
\newcommand{\green}[1]{{\color{green}#1}}
 
\newcommand{\bkt}[1]{\langle \mbox{ #1 } \rangle}
\newcommand{\br}{\mbox{~}|\mbox{~}}
 
\newcommand{\ns}[1]{\newpage\noindent {\bf #1}}

\setlength{\parindent}{0in}
 
\begin{document}
 
\huge

{\bf Recurrences}
\begin{align*}
f(1) &= 5\\
f(n) &= f(n-1) + 3  \tag{if $n>1$}
\end{align*}
Find some simple cases:
\begin{align*}
f(1) &= 5\\
f(2) &= f(1) + 3
 = 8\\
f(3) &= f(2) + 3
 = 11\\
f(4) &= f(3) + 3
 = 14
\end{align*}
Guess a solution:  $f(n) =  3n + 2$.  This is called a {\em
  closed form} since there is no recursion in the definition.

Check that it works on
the simple cases:
\begin{align*}
3(1) + 2 &= 5\\
3(2) + 2 &= 8\\
3(3) + 2 &= 11\\
3(4) +2 &= 14
\end{align*}
So it looks like we're on the right track.

How can you {\em prove} that it's the correct solution?

\ns{Proving recurrence solution by induction}
\begin{align*}
f(1) &= 5\\
f(n) &= f(n-1) + 3  \tag{if $n>1$}
\end{align*}

Prove $f(n) = 3n+2$ by
induction.  Base:
\[
f(1) = 5 = 
3(1) + 2
\]
So $f(n) = 3n+2$ is true when $n=1$.



Now assume \[f(n) = 3n + 2\] and prove \[f(n+1) = 3(n+1) + 2\]
Proof:
\begin{align*}
f(n+1) &= f(n) + 3\\
&= 3n+2 + 3\\
&= 3(n+1) + 2
\end{align*}
Guess checks out.  We have proved it is a correct solution by
induction.

\ns{Proof by induction for recurrence relations}

A simpler form of proof involves just plugging the closed form in for
$f$ and checking both clauses of the recurrence.
\begin{align*}
f(1) &= 5\\
f(n) &= f(n-1) + 3  \tag{if $n>1$}
\end{align*}
Plugging in $f(n) = 3n+2$ gives
\begin{align*}
3(1)+2 &= 5\\
3(n)+2 &= 3(n-1)+ 2 + 3  \tag{if $n>1$}
\end{align*}
and both of these check out, so it is proved by induction.


\ns{Solving recurrences without guessing}
\begin{align*}
f(1) &= 5\\
f(n) &= f(n-1) + 3 \tag{if $n>1$}
\end{align*}
Note that the second line represents infinitely many equations:
\begin{align*}
f(n) &= f(n-1) + 3\\
f(n-1) &= f(n-2) + 3\\
f(n-2) &= f(n-3) +3\\
f(n-3) &= f(n-4) + 3\\
\ldots\\
f(n-(k-1)) &= f(n-k) + 3\\
\ldots\\
f(n-(n-2)) &= f(n-(n-1)) + 3\\
&= f(1) + 3\\
&= 8
\end{align*}
As long as the argument to $f$ never goes below 1, all these are true.

\ns{Solving recurrences without guessing}
\begin{align*}
f(1) &= 5\\
f(n) &= f(n-1) + 3  \tag{if $n>1$}
\end{align*}
Let's {\bf add up} the left and right hand sides of 
some equations: 
\begin{align*}
f(n) &= f(n-1) + 3\\
f(n-1) &= f(n-2) + 3\\
f(n-2) &= f(n-3) +3\\
f(n-3) &= f(n-4) + 3
\end{align*}
Note that many of the terms appear on both sides, so we can subtract
them from the final equation:
\begin{align*}
f(n) &= \cancel{f(n-1)} + 3\\
\cancel{f(n-1)} &= \cancel{f(n-2)} + 3\\
\cancel{f(n-2)} &= \cancel{f(n-3)} +3\\
\cancel{f(n-3)} &= f(n-4) + 3
\end{align*}
We are left with:
\begin{align*}
f(n)
&=
f(n-4) +  3 +  + 3 +  3 + 3\\
&=f(n-4) +  4(3)
\end{align*}
If we added up 10 terms, we would get
\begin{align*}
f(n)
&=f(n-10) +  10(3)
\end{align*}

\ns{Adding up recurrence equations}
\begin{align*}
f(1) &= 5\\
f(n) &= f(n-1) + 3  \tag{if $n>1$}
\end{align*}
If we added up $k$ terms, we would get
\begin{align*}
f(n)
&=f(n-k) +  k(3)
\end{align*}
We can do this as long as $n-k \geq 1$.  If we let $n-k=1$, then
$k=n-1$ and we have
\begin{align*}
  f(n) &=
  f(n - (n - 1)) + (n-1)3\\
&= f(1) + (n-1)3\\
&= 5 + (n-1)3\\
&= 3n + 2
\end{align*}
Magic!

\ns{Let's see that again}
\begin{align*}
f(1) &= 5\\
f(n) &= f(n-1) + 3  \tag{if $n>1$}
\end{align*}
The recurrence gives us lots of equations:
\begin{align*}
f(n) &= \cancel{f(n-1)} + 3\\
\cancel{f(n-1)} &= \cancel{f(n-2)} + 3\\
\cancel{f(n-2)} &= \cancel{f(n-3)} +3\\
\cancel{f(n-3)} &= \cancel{f(n-4)} + 3\\
\ldots\\
\cancel{f(n-(n-2))} &= {f(n-(n-1))} + 3
\end{align*}
Add them all up and cancelling gives:
\begin{align*}
  f(n) &= f(1) + \sum_{k=1}^{n-1} 3\\
  &= 5 + (n-1)3 \\
  &= 3n + 2
  \end{align*}


\ns{Recurrences}
\begin{align*}
  f(1) &= 1\\
  f(n) &= 2f(n-1)  \tag{if $n>1$}
\end{align*}
Find some simple cases:
\begin{align*}
  f(1) &= 1\\
  f(2) &= 2f(1)= 2\\
  f(3) &= 2f(2)= 4\\
  f(4) &= 2f(3)= 8
\end{align*}
We can guess $f(n) = 2^{n-1}$. Trying some simple cases shows
this is a good guess:
\begin{align*}
  2^{1-1} &= 1\\
  2^{2-1} &= 2\\
2^{3-1} &= 4\\
2^{4-1} &= 8
\end{align*}
  We can now prove this guess correct
by induction.  Plug the closed form solution into the recursive equations:
\begin{align*}
  2^{(1-1)} &= 1\\
  2^{(n-1)} &= 2(2^{((n-1)-1)}) \tag{if $n>1$}
\end{align*}
And it checks out.

Now let's try to find that solution without guessing.

\ns{Recurrences solved without guessing}
\begin{align*}
  f(1) &= 1\\
  f(n) &= 2f(n-1)  \tag{if $n>1$}
\end{align*}
Get lots of equations:
\begin{align*}
  f(n) &= 2f(n-1)\\
  2f(n-1) &= 2(2(f(n-2)))\\
          &= 2^2f(n-2)\\
  2^2f(n-2) &= 2^2(2f(n-3))\\
  &= 2^3f(n-3)\\
  2^3f(n-3) &= 2^4f(n-4)\\
  &\ldots\\
  2^{k-1}f(n-(k-1)) &= 2^{k}f(n-k)\\
  &\ldots\\
  2^{n-2}f(2) &= 2^{n-1}f(1)
\end{align*}
We got that last one by considering the case where \[n-k=1\]
or, in other words, \[k = n-1\]

\ns{Add and cancel}
\begin{align*}
  f(1) &= 1\\
  f(n) &= 2f(n-1)  \tag{if $n>1$}
\end{align*}
\begin{align*}
  f(n) &= \cancel{2f(n-1)}\\
  \cancel{2f(n-1)} &= \cancel{2^2f(n-2)}\\
  \cancel{2^2f(n-2)} &= \cancel{2^3f(n-3)}\\
  \cancel{2^3f(n-3)} &= \cancel{2^4f(n-4)}\\
  &\ldots\\
  \cancel{2^{n-2}f(2)} &= 2^{n-1}f(1)
  \end{align*}
Using the fact that $f(1)=1$ gives 
\begin{align*}
  f(n)
  &= 2^{n-1}
\end{align*}
That's the same solution that we guessed in the beginning.


We already demonstrated that this solution works on small cases,
and we proved it works on all cases by induction.

\ns{A more tricky example}
\begin{align*}
  f(1) &= 5\\
  f(n) &= 3f(n-1) + 7
\end{align*}
Find some simple examples:
\begin{align*}
  f(1) &= 5\\
  f(2) &= 3f(1) + 7= 22\\
  f(3) &= 3f(2) + 7= 73\\
  f(4) &= 3f(3) + 7= 226
\end{align*}

What's the next number in the sequence?
\[
5,22,73,226,?
\]
Guessing may not be so helpful here.

\ns{Solve without guessing}
\begin{align*}
  f(1) &= 5\\
  f(n) &= 3f(n-1) + 7
\end{align*}
Get lots of equations:
\begin{align*}
  f(n) &= 3f(n-1) + 7\\
  3f(n-1) &= 3(3f(n-2) + 7)\\
          &= 3^2f(n-2) + 3(7)\\
  3^2f(n-2) &= 3^2(3f(n-3) + 7)\\
  &= 3^3f(n-3) + 3^2(7)\\
  3^3f(n-3) &= 3^4f(n-4) + 3^3(7)\\
  &\ldots\\
  3^{k-1}f(n-(k-1)) &= 3^kf(n-k) + 3^{k-1}(7)\\
  &\ldots\\
  {3^{n-2}f(2)} &= 3^{n-1}f(1) + 3^{n-2}(7)
\end{align*}


\ns{Solve without guessing}
\begin{align*}
  f(1) &= 5\\
  f(n) &= 3f(n-1) + 7
\end{align*}
Add and cancel:
\begin{align*}
  f(n) &= \cancel{3f(n-1)} + 7\\
  \cancel{3f(n-1)} &= \cancel{3^2f(n-2)} + 3(7)\\
  \cancel{3^2f(n-2)} &= \cancel{3^3f(n-3)} + 3^2(7)\\
  \cancel{3^3f(n-3)} &= \cancel{3^4f(n-4)} + 3^3(7)\\
  &\ldots\\
  \cancel{3^{k-1}f(n-(k-1))} &= \cancel{3^kf(n-k)} + 3^{k-1}(7)\\
  &\ldots\\
  \cancel{3^{n-2}f(2)} &= 3^{n-1}f(1) + 3^{n-2}7
  \end{align*}
We are left with
\begin{align*}
  f(n) &= 3^{n-1}(5) + \sum_{i=0}^{n-2} 3^i(7)\\
  &= 5(3^{n-1}) + 7\sum_{i=0}^{n-2} 3^i
\end{align*}

\ns{Solve the summation}
\begin{align*}
  f(1) &= 5\\
  f(n) &= 3f(n-1) + 7
\end{align*}
We have a geometric series, so
\begin{align*}
  f(n)   &=5(3^{n-1}) + 7\sum_{i=0}^{n-2} 3^i\\
  &= 5(3^{n-1}) + 7\frac{3^{n-1} -1}{3-1}\\\\
&= \frac{10(3^{n-1}) + 7(3^{n-1}) - 7}{2}\\\\
  &= \frac{17(3^{n-1}) - 7}{2}
\end{align*}
Seriously?

\ns{Check our solution}
\begin{align*}
  f(1) &= 5\\
  f(n) &= 3f(n-1) + 7
\end{align*}
Closed form:
\begin{align*}
  f(n)  
  &= \frac{17(3^{n-1}) - 7}{2}
\end{align*}
Remember our sequence?
\[
5,22,73,226,?
\]
Let's test $f(4)=226$:
\begin{align*}
  f(4) &= \frac{17(3^{4-1}) - 7}{2}\\
  &= \frac{17(27) - 7}{2}\\
  &= \frac{459 - 7}{2}\\
  &= \frac{452}{2}\\
  &= 226
\end{align*}
Wow.

\bigskip

(Next number is 685, by the way.  Obvious, no?)

\ns{Prove correct by induction}
\begin{align*}
  f(1) &= 5\\
  f(n) &= 3f(n-1) + 7
\end{align*}
Closed form:
\begin{align*}
  f(n)  
  &= \frac{17(3^{n-1}) - 7}{2}
\end{align*}
Base:
\begin{align*}
  \frac{17(3^{1-1}) - 7}{2}  &\stackrel{?}{=} 5 \tag{Correct}
\end{align*}
Step:
\begin{align*}
  \frac{17(3^{n-1}) - 7}{2}
  &\stackrel{?}{=}
  3 \left(\frac{17(3^{n-2}) - 7}{2}\right) + 7\\\\
  &=\frac{17(3^{n-1}) - 3(7) + 14}{2}\\\\
    &=\frac{17(3^{n-1}) - 7}{2}\tag{Correct}
\end{align*}
We have proved by induction that our closed form is a solution to the
original recurrence.
\end{document}





