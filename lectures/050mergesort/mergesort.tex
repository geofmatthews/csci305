\documentclass{article}
\usepackage{fullpage}

\usepackage{amsmath,cancel}
\usepackage{graphicx}
\usepackage{color}
\usepackage{alltt}
\newcommand{\red}[1]{{\color{red}#1}}
\newcommand{\cyan}[1]{{\color{cyan}#1}}
\newcommand{\blue}[1]{{\color{blue}#1}}
\newcommand{\magenta}[1]{{\color{magenta}#1}}
\newcommand{\yellow}[1]{{\color{yellow}#1}}
\newcommand{\green}[1]{{\color{green}#1}}
 
\newcommand{\bkt}[1]{\langle \mbox{ #1 } \rangle}
\newcommand{\br}{\mbox{~}|\mbox{~}}
 
\newcommand{\ns}[1]{\newpage {\bf #1}}

\setlength{\parindent}{0in}
 
\begin{document}
 
\huge


{\bf Mergesort Recurrence}
\begin{align*}
T(1) &= c_1\\
T(n) &= 2T(n/2) + cn  \tag{if $n>1$}
\end{align*}
Get lots of equations:
\begin{align*}
  T(n) &= 2T(n/2) + cn  \\
  2T(n/2) &= 2[2T(n/2^2) + cn/2]\\
          &= 2^2T(n/2^2) + cn\\
 2^2T(n/2^2) &= 2^2[2T(n/2^3) + cn/2^2]\\
             &= 2^3T(n/2^3) + cn\\
 &\ldots\\
 2^{k-1}T(n/2^{k-1}) &= 2^kT(n/2^k) + cn\\
 &\ldots\\
 2^{\lg n - 1}T(n/2^{\lg n - 1}) &= 2^{\lg n}T(n/2^{\lg n}) + cn\\
 \end{align*}

\ns{Mergesort Recurrence}

\begin{align*}
T(1) &= c_1\\
T(n) &= 2T(n/2) + cn  \tag{if $n>1$}
\end{align*}
Rewrite equations, sum and cancel:

\begin{align*}
  T(n) &= \cancel{2T(n/2)} + cn  \\
  \cancel{2T(n/2)} 
          &= \cancel{2^2T(n/2^2)} + cn\\
 \cancel{2^2T(n/2^2)}
             &= \cancel{2^3T(n/2^3)} + cn\\
 &\ldots\\
 \cancel{2^{k-1}T(n/2^{k-1})} &= \cancel{2^kT(n/2^k)} + cn\\
 &\ldots\\
 \cancel{2^{\lg n - 1}T(n/2^{\lg n - 1})}
 &= 2^{\lg n}T(n/2^{\lg n}) + cn\\
 \end{align*}
Which results in
\begin{align*}
  T(n) &= 2^{\lg n}T(N/2^{\lg n}) + \sum_{k=1}^{\lg n} cn\\
  &= c_1n + cn\lg n\\
  &= \Theta(n\lg n)
\end{align*}



\ns{Mergesort Recurrence, changing function}

\begin{align*}
T(1) &= c\\
T(n) &= 2T(n/2) + cn  \tag{if $n>1$}
\end{align*}
Let's make this easier by assuming $n$ is a power of two:
\begin{align*}
  n &= 2^j\\
  j &= \log_2(n)
\end{align*}
Now we can rewrite our equations:
\begin{align*}
T(2^0) &= c\\
T(2^j) &= 2T(2^{j-1}) + cn  \tag{if $j>0$}
\end{align*}
Now we can introduce a new function, $f(j) = T(2^j)$, and we have
\begin{align*}
  f(0) &= c\\
  f(j) &= 2f(j-1) + c2^j
\end{align*}
Let's try to solve this recurrence.

\ns{Solve the recurrence}
\begin{align*}
  f(0) &= c\\
  f(j) &= 2f(j-1) + c2^j
\end{align*}
Get lots of equations
\begin{align*}
  f(j) &= 2f(j-1) + c2^j\\
  2  f(j-1) &= 2^2f(j-2) + c2(2^{j-1})\\
  &=  2^2f(j-2) + c2^{j}\\
  2^2f(j-2) &= 2^3f(j-3) + c2^2(2^{j-2})\\
  &= 2^3f(j-3) + c2^{j}\\
  &\ldots\\
  2^{k-1}f(j-(k-1)) &= 2^kf(j-k) + c2^j\\
  &\ldots\\
  2^{j-1}f(j-(j-1)) &= 2^jf(j-j) + c2^j\\
2^{j-1}f(j-1)  &= c2^j + c2^j
\end{align*}

\ns{Simplify and cancel}
\begin{align*}
  f(0) &= c\\
  f(j) &= 2f(j-1) + c2^j
\end{align*}
Cancelling:
\begin{align*}
  f(j) &= \cancel{2f(j-1)} + c2^j\\
  \cancel{2  f(j-1)}
  &=  \cancel{2^2f(j-2)} + c2^{j}\\
 \cancel{ 2^2f(j-2)}
  &= \cancel{2^3f(j-3)} + c2^{j}\\
  &\ldots\\
  \cancel{2^{k-1}f(j-(k-1))} &= \cancel{2^kf(j-k)} + c2^j\\
  &\ldots\\
  \cancel{2^{j-1}f(1)}
  &= c2^j + c2^j
\end{align*}
Gives
\begin{align*}
  f(j) &= c2^j + \sum_{k=1}^{j} c2^j\\
  &= (j+1)c2^j
\end{align*}

\ns{Remembering $f(j) = T(2^j)$}
\begin{align*}
T(1) &= c\\
T(n) &= 2T(n/1) + cn  \tag{if $n>1$}\\\\
n &=  2^j \\
  j &= \log_2(n)\\\\
  f(0) &= c\\
  f(j) &= 2f(j-1) + c2^j\\\\
  f(j)  &= (j+1)c2^j\\\\
  T(n) &= T(2^j)\\
  &= f(j)\\
  &= (j+1)c2^{j}\\
  &= (\log_2(n) + 1)c2^{\log_2(n)}\\
  &= cn\log_2(n) + cn\\
  &= \Theta(n\lg(n))
\end{align*}


\end{document}





