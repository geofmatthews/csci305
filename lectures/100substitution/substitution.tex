\documentclass{article}
\usepackage{fullpage}

\usepackage{amsmath,cancel}
\usepackage{graphicx}
\usepackage{color}
\usepackage{alltt}
\newcommand{\red}[1]{{\color{red}#1}}
\newcommand{\cyan}[1]{{\color{cyan}#1}}
\newcommand{\blue}[1]{{\color{blue}#1}}
\newcommand{\magenta}[1]{{\color{magenta}#1}}
\newcommand{\yellow}[1]{{\color{yellow}#1}}
\newcommand{\green}[1]{{\color{green}#1}}
 
\newcommand{\bkt}[1]{\langle \mbox{ #1 } \rangle}
\newcommand{\br}{\mbox{~}|\mbox{~}}
 
\newcommand{\ns}[1]{\newpage {\bf #1}}

\newcommand{\bi}{\begin{itemize}}
\newcommand{\ii}{\item}
\newcommand{\ei}{\end{itemize}}

\setlength{\parindent}{0in}
 
\begin{document}

 
\LARGE

\newpage

{\bf Equations and recurrences}
\bi
\ii
Let's start backwards.
Suppose we know what the function is, for example:
\begin{align*}
  f(n) &= 3n^2 + 5
\end{align*}
\ii
We want to get a recurrence for this function,
so let's try some of our favorite recurrences.
\ii
First, $f(n-1)$:
\begin{align*}
  f(n-1) &= 3(n-1)^2 + 5\\
  &= 3n^2 -6n + 3 + 5\\
  &= (3n^2 + 5) -6n + 3\\
  &= f(n) -6n + 3
\end{align*}
\ii
And now we have it!  Our original function is the solution to
the recurrence:
\begin{align*}
  f(n) &= f(n-1) + 6n - 3
\end{align*}
\ii
We can check this by substituting the original function in both sides
of this:
\begin{align*}
  3n^2 + 5 &= 3(n-1)^2 + 5 + 6n - 3\\
  &= 3n^2 - 6n + 3 + 5 + 6n - 3\\
  &= 3n^2 + 5
  \end{align*}

\ei

\newpage

{\bf Equations and recurrences}
\bi
\ii
Let's try a different recurrence for the same function.
\begin{align*}
  f(n) &= 3n^2 + 5
\end{align*}
\ii
Let's try , $f(n/2)$:
\begin{align*}
  f(n/2) &= 3(n/2)^2 + 5\\
  &= (3/4)n^2 + 5\\
  &= 3n^2 + 5 - (9/4)n^2 \\
  &= f(n) - (9/4)n^2
\end{align*}
\ii
And now we have it!  Our original function is the solution to
the recurrence:
\begin{align*}
  f(n) &= f(n/2) + (9/4)n^2
\end{align*}
\ii
Now let's check this with our original equation plugged into both
sides:
\begin{align*}
  3n^2 + 5 &= 3(n/2)^2 + 5 + (9/4)n^2\\
  &= (3/4)n^2 + 5 + (9/4)n^2\\
  &= 3n^2 + 5
\end{align*}
\ii
So, again, we have verified that our original function is a solution
to this recurrence.

\ei
\newpage

{\bf Substitution method, big-$O$}
\bi
\ii
Now let's go the other way.  Suppose we have an innocent recurrence,
like this one:
\begin{align*}
  f(n) &= f(n-1) + 6n - 3
\end{align*}
\ii
And we want to solve it.  (Don't tell anybody we already know the
solution.)
\ii
And suppose that we suspect the solution is $O(n^2)$.
How can we \textit{prove} this?  
\ii
Put in other words, can we prove by induction that there exists some
$c$ such that $f(n) \leq cn^2$ for sufficiently large $n$?
Can we prove that this is a solution by induction?
\ii
The base case is trivial, since we just have to show there exists some
$c$ such that $f(1)\leq c$, which will always be true.
\ei

\newpage

{\bf Substitution method, big-$O$, inductive step}
\bi
\ii Now for the inductive step.  
\begin{align*}
  f(n) &= f(n-1) + 6n - 3
\end{align*}
\ii
Since our recurrence is on $f(n-1)$, we
will assume, by inductive hypothesis, that $f(n-1) \leq c(n-1)^2$ and
on the basis of this try to prove that $f(n) \leq cn^2$.
\begin{align*}
  f(n) &= f(n-1) + 6n - 3\\
  &\leq c(n-1)^2 + 6n - 3\\
  &= cn^2 -2cn + c + 6n - 3\\
  &= cn^2 + (6 -2c)n + (c - 3)\\
  &\leq cn^2
\end{align*}
\ii
So long as
\begin{align*}
  (6 -2c)n + (c - 3) &\leq 0\\
  (6 -2c)n   &\leq 3-2\\
  n   &\leq \frac{3-2}{6 -2c}\\
  n   &\geq \frac{1}{2c-6}\\
\end{align*}
which we can use in our definition of ``sufficiently large'' $n$, so
this checks out, too.

\ei
  
    
\newpage

{\bf Substitution method, big-$\Omega$, inductive step}
\bi
\ii Now let's try that method for $\Omega$.
\begin{align*}
  f(n) &= f(n-1) + 6n - 3
\end{align*}
\ii Suppose $f(n-1)\geq c(n-1)^2$ and prove
$f(n)\geq cn^2$ 
\begin{align*}
  f(n) &= f(n-1) + 6n - 3\\
  &\geq c(n-1)^2 + 6n - 3\\
  &= cn^2 -2cn + c + 6n - 3\\
  &= cn^2 + (6 -2c)n + (c - 3)\\
  &\geq cn^2
\end{align*}
So long as $2c<6$, and sufficiently large $n$.

\ei


\newpage
{\bf Substitution method}
\begin{align*}
  T(1) &= c\\
  T(n) &= 8T(n/2) + \Theta(n^2)
\end{align*}
\begin{itemize}
  \item
We guess that $T(n) = O(n^3)$ and prove it true by induction.
\item
Base case is some constant, which can always be chosen large enough.
\item
Induction, we need to show there is some constant
$c_0$ such that
if we assume $T(m) \leq c_0m^3$ for all $m<n$ (strong induction),
we can then prove $T(n) \leq c_0n^3$ (same constant).
\item
First try:
\begin{align*}
  T(n) &= 8T(n/2) + \Theta(n^2)&\mbox{use definition of $\Theta$}\\
  &\leq  8T(n/2) + cn^2 & n > n_0\\
  &\leq 8c_0n^3/2^3 +  cn^2 & \mbox{since $n/2 < n$}\\
  &= c_0n^3 +  cn^2\\
  &\stackrel{?}{\leq} c_0n^3
\end{align*}
\item
  We can't really prove what we want.
\item
  We can't guarantee it is smaller than our target, because we don't know
  what $c$ is.
\item
  Be careful!  Don't be tempted to say
  $cn^2$ doesn't matter in a big-$0$ proof.
\item
  It {\em does}
  here because this is just one step
in an inductive proof.
\end{itemize}

\newpage
{\bf Substitution method, second try}
\begin{align*}
  T(1) &= c\\
  T(n) &= 8T(n/2) + \Theta(n^2)
\end{align*}
\begin{itemize}
  \item
We guess that $T(n) = O(n^3)$ and prove it true by induction.
\item
Base case is some constant, which can always be chosen large enough.
\item
We use a different member of $O(n^3)$ for induction.
\item
  We need to show that there exist some constantes
$c_0,c_1$ such that
if we assume $T(m) \leq c_0m^3-c_1m^2$ for all $m<n$ (strong induction),
we can then prove $T(n) \leq c_0n^3-c_1n^2$ (same constants).  
\begin{align*}
  T(n) &= 8T(n/2) + \Theta(n^2)&\mbox{use definition of $\Theta$}\\
  &\leq  8T(n/2) + cn^2 & n > n_0\\
  &\leq 8(c_0n^3/2^3 - c_1n^2/2^2)+  cn^2 & \mbox{since $n/2 < n$}\\
  &= c_0n^3 -  2c_1n^2 + cn^2\\
  &= c_0n^3 - c_1n^2 &\mbox{choose $c_1 = c$}
\end{align*}
\end{itemize}

\end{document}





