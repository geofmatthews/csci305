\documentclass{beamer}

\usepackage{amsmath,cancel}
\usepackage{graphicx}
\usepackage{color}
\usepackage{alltt}

\newcommand{\sect}[1]{
\section{#1}
\begin{frame}[fragile]\frametitle{#1}
}

\newcommand{\bi}{\begin{itemize}}
\newcommand{\ii}{\item}
\newcommand{\ei}{\end{itemize}}

\begin{document}
\sect{Finding $\Theta$ for Recurrences}

\bi
\ii We used an ``iterate and cancel'' method to solve recurrences.
\ii The textbook does not discuss solving recurrences.
\ii However, it discusses three methods for finding big-$\Theta$ for
recurrences, which is good enough for our work.
\ii The first is the substitution method.
\bi
\ii It is basically ``guess and check.''
\ii But easier because you only have to guess big-$\Theta$.
\ei
\ii The second is the recursion tree method.
\bi
\ii This is essentially a pictorial match to our iterative method.
\ii It may help generating guesses, but definitely needs to be proved
using induction.
\ei
\ii The third is the master theorem.
\ei

\end{frame}

\sect{Substitution Method}
\bi
\ii
Prove that $T(n) = O(n\lg n)$ for the following recurrence:
\begin{align*}
T(n) &= 2T(n/2) + n
\end{align*}
\ii
Basically we guess the form of the element of
$O(n\lg n)$, up to some constants,
and then try to prove it works by induction.
\ii
There are a number of possible guesses:
\begin{align*}
  T(n) &\leq cn\lg n\\
  T(n) &\leq cn\lg n + dn\\
  T(n) &\leq cn\lg n + d\lg n\\
  T(n) &\leq cn\lg n + d\lg n + b n\\
  T(n) &\leq cn\lg n + d\sqrt{n}
\end{align*}
\ii
We have to hope we hit on the right guess fairly quickly.
\ii
Working with one guess can inform the next guess.
\ei

\end{frame}

\sect{Substitution Method}
\bi
\ii
Prove that $T(n) = O(n\lg n)$ for the following recurrence:
\begin{align*}
T(n) &= 2T(n/2) + n
\end{align*}
\ii
Guess that $T(n) \leq cn\lg n$ for some $c$ and prove it by induction.
\ii
Assume that $T(x) \leq cx\lg x$ for all $x < n$ and prove:
\begin{align*}
  T(n) &= 2T(n/2) + n & \mbox{by recurrence}\\
  &\leq 2(c(n/2)\lg(n/2)) + n &\mbox{by inductive hypothesis}\\
  &= cn\lg(n/2) + n\\
  &= cn\lg n - cn\lg 2 + n\\
  &= cn\lg n - cn + n\\
  &\leq cn\lg n &\mbox{for $c \geq 1$}
\end{align*}
\ii
Can also prove $T(n) = \Omega(n\lg n)$ the same way.
\ii
So, $T(n) = \Theta(n \lg n)$.
\ei
\end{frame}

\sect{Base case}
\bi
\ii Note that we didn't bother with the base case for induction.
\ii We can always choose a constant such that it's bigger than $T(1)$,
which is also a constant.
\ei
\end{frame}

\sect{Substitution method, guessing wrong element of $O(n)$}
\bi
\ii
Prove that $T(n) = O(n)$ for
\begin{align*}
  T(n) &= 2T(n/2) + 1
\end{align*}
\ii
Guess that $T(n) \leq cn$.
\ii
Assume that $T(x) \leq cx$ for $x < n$.
\begin{align*}
  T(n) &= 2T(n/2) + 1 \\
       &\leq 2(cn/2) + 1\\
       &= cn + 1
\end{align*}
\ii
This is {\em not} good enough, even though $cn +1 = O(n)$.
\ii
The problem is this is {\em one} step in an inductive proof,
and if we add 1 for every step from 1 to $n$, we get more than $O(1)$.
\ei
\end{frame}
\sect{Try a more general guess}
\bi
\ii
Prove that $T(n) = O(n)$ for
\begin{align*}
  T(n) &= 2T(n/2) + 1
\end{align*}
\ii
Guess that $T(n) \leq cn + d$.
\ii
Assume $T(x) \leq cx + d$ for $x < n$.
\begin{align*}
  T(n) &= 2T(n/2) + 1 \\
       &\leq 2(cn/2 + d) + 1\\
       &= cn + 2d + 1\\
       &\leq cn  & d < -1
\end{align*}
\ii
This works, because we are free to pick any value for $c$ and $d$!
\ii
Note that we must pick {\em one} value for $c$ and $d$ for
all inductive steps.  
\ei
\end{frame}

\sect{Changing Variables}
\bi
\ii
Some very difficult recurrences can be solved simply
with change of variables.
\ii Example:
\begin{align*}
  T(n) &= 2T(\sqrt{n}) + \lg n
\end{align*}
\ii
Use the following substitutions:
\begin{align*}
  m &= \lg n\\
  T(2^m) &= 2T(2^{m/2}) + m\\
  S(m) &= T(2^m)\\
  S(m) &= 2S(m/2) + m
\end{align*}
\ii
This can be solved to show that $S(m) =O(m\lg m)$, therefore
\[
T(n) = T(2^m) = S(m) =O(m \lg m) = O(\lg n \lg \lg n)
\]
\ei
\end{frame}



%\sect{Another example}
%\bi
%\ii
%Show that $T(n) = O(n^3)$ for
%\begin{align*}
%  T(n) &= 8T(n/2) + \Theta(n^2)
%\end{align*}
%\ii
%Assume $T(x) \leq cx^3$ for all $x<n$ (strong induction).
%\ii
%Try to prove $T(n) \leq cn^3$ (same constant).
%\begin{align*}
%  T(n) &= 8T(n/2) + \Theta(n^2)\\
%  &\leq  8T(n/2) + an^2  &\mbox{definition of $\Theta$}\\
%  &\leq 8cn^3/2^3 +  an^2 & \mbox{since $n/2 < n$}\\
%  &= cn^3 +  an^2
%\end{align*}
%\ii
%We cannot use $cn^3 +  an^2 = O(n^3)$.
%\ii
%We need a more general hypothesis.
%\ei
%\end{frame}
%
%\sect{Substitution method, second try}
%\bi
%\ii
%Show that $T(n) = O(n^3)$ for
%\begin{align*}
%  T(n) &= 8T(n/2) + \Theta(n^2)
%\end{align*}
%\ii
%Assume $T(x) \leq cx^3 + dx^2$ for all $x<n$.
%\ii
%Try to prove $T(n) \leq cn^3 + dx^2$.
%\begin{align*}
%  T(n) &= 8T(n/2) + \Theta(n^2)\\
%  &\leq  8T(n/2) + an^2  &\mbox{definition of $\Theta$}\\
%  &\leq 8(cn^3/2^3 + dn^2/2^3) +  an^2 & \mbox{inductive hypothesis}\\
%  &= cn^3 +  dn^2 + an^2 &\mbox{let $d = -a$}\\
%  &= cn^3 \leq cn^3 + dn^2
%\end{align*}
%\ii
%Note that there is only {\em one} $\Theta(n^2)$ function in question.
%\ii
% Therefore
%there is only one constant, $a$, for all recurrence equations.
%\ei
%\end{frame}


\sect{The Jeopardy Method}

\bi
\ii
In order to practice solving recurrences, we need a lot of
examples.
\ii
It's difficult to know in advance whether a recurrence has
a simple solution.
\ii
We can solve this (tactical) problem by inventing our
own recurrences, by starting with the {\em solution} to
the recurrence (a function definition) and then discovering
what recurrence it is a solution to.
\ii
We can call this the ``Jeopardy Method'' of finding recurrences.
\ei
\end{frame}

\sect{The Jeopardy Method}
\bi
\ii
Let's start with this function, for example:
\begin{align*}
  f(n) &= 3n^2 + 5
\end{align*}
\ii
We know the base case
\[
f(1) = 8
\]
\ii To get a recursion, do some simple math:
\begin{align*}
  f(n-1) &= 3(n-1)^2 + 5\\
  &= 3n^2 -6n + 3 + 5\\
  &= (3n^2 + 5) -6n + 3\\
  &= f(n) -6n + 3
\end{align*}
\ii
Therefore, our original function is the solution to
the recurrence:
\begin{align*}
  f(1) &= 8\\
  f(n) &= f(n-1) + 6n - 3
\end{align*}
\ei
\end{frame}
\sect{Check your question to the solution}
\bi
\ii
We can check this by substituting the function in both sides
of the recurrence:
\begin{align*}
  f(n) &= 3n^2 + 5\\\\
  f(n) &= f(n-1) + 6n - 3
  \\\\
  3n^2 + 5 &= 3(n-1)^2 + 5 + 6n - 3\\
  &= 3n^2 - 6n + 3 + 5 + 6n - 3\\
  &= 3n^2 + 5
  \end{align*}

\ei



\end{frame}
\sect{Equations and recurrences}
\bi
\ii
Let's try a different recurrence for the same function.
\begin{align*}
  f(n) &= 3n^2 + 5
\end{align*}
\ii
Let's try, $f(n/2)$:
\begin{align*}
  f(n/2) &= 3(n/2)^2 + 5\\
  &= (3/4)n^2 + 5\\
  &= 3n^2 + 5 - (9/4)n^2 \\
  &= f(n) - (9/4)n^2
\end{align*}
\ii
Our original function is also the solution to
the recurrence:
\begin{align*}
  f(1) &= 8\\
  f(n) &= f(n/2) + (9/4)n^2
\end{align*}
\ei
\end{frame}
\sect{Check your solution}
\bi
\ii
Now let's check this with our original equation plugged into both
sides:
\begin{align*}
  f(n) &= 3n^2 + 5
  \\\\
  f(n) &= f(n/2) + (9/4)n^2
\\\\
  3n^2 + 5 &= 3(n/2)^2 + 5 + (9/4)n^2\\
  &= (3/4)n^2 + 5 + (9/4)n^2\\
  &= 3n^2 + 5
\end{align*}
\ii
So, again, we have verified that our original function is a solution
to this recurrence.

\ei

\end{frame}



\sect{Substitution method, random example}
\bi
\ii
Show that $f(n) = O(n^2)$.
\begin{align*}
  f(n) &= f(n-1) + 6n - 3
\end{align*}
\ii
Assume, by inductive hypothesis, that $f(n-1) \leq c(n-1)^2$.
\ii
Try to prove that $f(n) \leq cn^2$.
\begin{align*}
  f(n) &= f(n-1) + 6n - 3\\
  &\leq c(n-1)^2 + 6n - 3\\
  &= cn^2 -2cn + c + 6n - 3\\
  &= cn^2 + (6 -2c)n + (c - 3)
\end{align*}
\ii
Now we just have to pick $c$ and $n_0$ such that
\begin{align*}
(6 -2c)n + (c - 3)  &\leq 0
\end{align*}
\ei
\end{frame}

\sect{Picking $c$ and $n_0$}
\bi
\ii
We want to assure
\begin{align*}
(6 -2c)n + (c - 3)  &\leq 0
\end{align*}
\ii
If we pick $c > 3$ then
\begin{align*}
  (6 - 2c) &< 0 
\end{align*}
\ii
Therefore
\begin{align*}
(6 -2c)n + (c - 3)  \leq 0  &\Longleftrightarrow
  n \geq \frac{(3-c)}{(6-2c)}
\end{align*}
\ii
So, let
\begin{align*}
n_0 &= \left\lceil\frac{(3-c)}{(6-2c)}\right\rceil
\end{align*}
\ei
\end{frame}


\sect{Substitution method, big-$\Omega$}
\bi
\ii To prove $\Theta(n^2)$ we also need $\Omega(n^2)$.
\begin{align*}
  f(n) &= f(n-1) + 6n - 3
\end{align*}
\ii
Suppose $f(n-1)\geq c(n-1)^2$ and prove
$f(n)\geq cn^2$ 
\begin{align*}
  f(n) &= f(n-1) + 6n - 3\\
  &\geq c(n-1)^2 + 6n - 3\\
  &= cn^2 -2cn + c + 6n - 3\\
  &= cn^2 + (6 -2c)n + (c - 3)\\
  &\geq cn^2
\end{align*}
\ii
So long as $c<3$, and sufficiently large $n$.
\ii
Since $f(n) = O(n^2)$ and $f(n) = \Omega(n^2)$, $f(n) = \Theta(n^2)$.
\ei
\end{frame}

\end{document}





