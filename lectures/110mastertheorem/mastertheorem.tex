\documentclass{beamer}
\usepackage{cancel}
\usepackage{fancyvrb}
\usepackage{hyperref}
\usepackage{alltt}
\usepackage{graphicx}
\newcommand{\sect}[1]{
\section{#1}
\begin{frame}[fragile]\frametitle{#1}
}

\newtheorem{theo}{Theorem}[section]

\newcommand{\myfig}[1]{}%{\centerline{\includegraphics[scale=0.25]{figures/#1.png}}}
\newcommand{\myfigt}[1]{}%{\centerline{\includegraphics[scale=0.2]{figures/#1.png}}}
\newcommand{\myfigg}[1]{}%{\centerline{\includegraphics[scale=0.125]{figures/#1.png}}}

\newcommand{\trans}[5]{
\begin{tabular}{|c|c|c|c|c|}\hline
#1 & #2 & #3 & #4 & #5 \\\hline
\end{tabular}
}

\newcommand{\arr}{&\rightarrow&}
\newcommand{\darr}{&\Rightarrow&}
\newcommand{\ar}{\ensuremath{\rightarrow}}
\newcommand{\dar}{\ensuremath{\Rightarrow}}
\newcommand{\bee}{\begin{eqnarray*}}
\newcommand{\eee}{\end{eqnarray*}}
\newcommand{\lmb}{\ensuremath{\lambda}}

\newcommand{\bi}{\begin{itemize}}
\newcommand{\li}{\item}
\newcommand{\ei}{\end{itemize}}


\newcommand{\lrtable}{
\begin{tabular}{lrlr}
Stack & Input & Rule & Peek\\\hline
 $\$ aCd$ & $\$$&S\ar aCd & \\
 $\$ ac^+C$ & $d \$$ & C \ar cC & \\
 $\$ ac^+$&$d \$$ & C \ar c & d\\
 $\$ bCD$&$ \$$ & S\ar bCD & \\
 $\$ bCd$& $ \$$& D\ar d & \\
 $\$ bc^+C$&$d \$$ & C \ar cC & \\
 $\$ bc^+$&$d \$$& C \ar c& d\\\hline
\end{tabular}
}


\mode<presentation>
{
%  \usetheme{Madrid}
  % or ...

%  \setbeamercovered{transparent}
  % or whatever (possibly just delete it)
}

\usepackage[english]{babel}

\usepackage[latin1]{inputenc}

\title[Notes on Recursion Trees]
{
 Notes on Recursion Trees
}

\subtitle{} % (optional)

\author[Geoffrey Matthews]
{Geoffrey Matthews}
% - Use the \inst{?} command only if the authors have different
%   affiliation.

\institute[WWU/CS]
{
  Department of Computer Science\\
  Western Washington University
}
% - Use the \inst command only if there are several affiliations.
% - Keep it simple, no one is interested in your street address.

\date{\today}

% If you have a file called "university-logo-filename.xxx", where xxx
% is a graphic format that can be processed by latex or pdflatex,
% resp., then you can add a logo as follows:

%\pgfdeclareimage[height=0.5cm]{university-logo}{WWULogoProColor}
%\logo{\pgfuseimage{university-logo}}

% If you wish to uncover everything in a step-wise fashion, uncomment
% the following command: 

%\beamerdefaultoverlayspecification{<+->}

\begin{document}

%\begin{frame}
%  \titlepage
%\end{frame}


\newcommand{\myref}[1]{\small\item\url{#1}}
\newcommand{\myreft}[1]{\footnotesize\item\url{#1}}

%\begin{frame}
%  \frametitle{Outline}
%  \tableofcontents
%  % You might wish to add the option [pausesections]
%\end{frame}

\sect{Master Theorem}
Let $a\geq 1$ and $b > 1$ be constants, let $f(n)$ be a function, and
let $T(n)$ be defined on nonnegative integers by
\[
T(n) = aT\left(\frac{n}{b}\right) + f(n)
\]
Then $T(n)$ has the following asymptotic bounds:
\bigskip
\begin{center}
{\large
\begin{tabular}{r|l}
$f(n)$ & $T(n)$ \\\hline 
$\mbox{O}(n^{\log_b a - \epsilon})$& $ \Theta(n^{\log_b a })$\\
$\Theta(n^{\log_b a})$ & $\Theta(f(n)\lg n)$\\
$\Omega(n^{\log_b a + \epsilon})$ & $\Theta(f(n))$\\
\end{tabular}
}\end{center}
\bigskip

The last one only if $af(\frac{n}{b}) \leq cf(n)$ for some $c<1$
and large enough $n$.
\end{frame}

\sect{Master theorem does not apply}
\begin{align*}
  T(n) &= 2T(n/2) + n\lg n\\
  n^{\log_b a} &= n\\
  f(n) &= n\lg n\\
  f(n) &= \Omega(n^{\log_b a}) = \Omega(n)\\
  f(n) &\not= \Omega(n^{\log_b a+\epsilon}) = \Omega(n^{1+\epsilon})\\
n\lg n &\not= \Omega(n^{1.0000000000001})\\
\lg n &\not= \Omega(n^{0.0000000000001})\\
\end{align*}
\end{frame}

\sect{Master theorem does apply}
\begin{align*}
  T(n) &= 4T(n/3) + n\lg n\\
  n^{\log_b a} &= n^{\log_c 4} = n^{1.26...}\\
  f(n) &= n\lg n\\
  f(n) &= O(n^{\log_b a}) = O(n^{1.26...})\\
  f(n) &= O(n^{\log_b a-\epsilon}) = O(n^{1.2}) & \epsilon = 0.06...\\
  n\lg n &= O(n^{1.2})\\
  \lg n &= O(n^{0.2})\\
\end{align*}
\end{frame}
\end{document}
