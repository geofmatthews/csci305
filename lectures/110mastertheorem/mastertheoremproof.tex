\documentclass{article}
\usepackage[margin=1in]{geometry}
\usepackage{graphicx}
\usepackage{amsmath}
\title{Proof of Master Theorem}
\author{Geoffrey Matthews}
\begin{document}
\maketitle
\centerline{\Large $T(n) = 3T(n/4) + cn^2$}
\hrulefill

\includegraphics[width=\textwidth]{Fig-4-5.pdf}


\section{Divide and Conquer Recurrence}
Suppose that $a\geq 1$ and $b\geq 1$ are constants, let $f(n)$ be a
function, and let $T(n)$ be defined on natural numbers with
the recurrence:
\LARGE
\begin{align}
T(n) = aT\left(\frac{n}{b}\right) + f(n)
\end{align}
\normalsize
Let's assume $n$ is a power of $b$, so that
\LARGE
\begin{align}
n &= b^k\\
k&=\log_b(n)\end{align}
\normalsize
Now 
let's see what we get by trying to spell out terms and then cancel:
\LARGE
\begin{align}
T(b^k) &= aT(b^{k-1}) + f(b^k)\\
aT(b^{k-1})
&=
a^2T(b^{k-2}) + af(b^{k-1})
\\
a^2T(b^{k-2})
&=
a^3T(b^{k-3}) + a^2f(b^{k-2})
\\
&\ldots
\\
a^{k-1}T(b^{1})
&=
a^kT(1) + a^{k-1}f(b)
\end{align}
\normalsize
Adding up both sides while cancelling terms, we get
\LARGE
\begin{align}
T(b^k) &= a^kT(1) + \sum_{i=1}^{k} a^{k-i}f(b^{i})
\\
&=
a^{\log_b(n)}T(1) + \sum_{i=1}^{\log_b(n)} a^{k-i}f(b^{i})
\\
&=
n^{\log_b(a)}T(1) + \sum_{i=1}^{\log_b(n)} a^{k-i}f(b^{i}) \label{tcomplexity}
\end{align}
\normalsize
Since $T(1)$ will be a constant, the complexity of the whole thing
depends on whether 
\LARGE
\begin{align} n^{\log_b(a)}\label{base}\end{align}
\normalsize
grows faster or slower than
\LARGE
\begin{align}
  \sum_{i=1}^{\log_b(n)} a^{k-i}f(b^{i}) \label{expression1}
\end{align}
\normalsize
Note that expression \ref{expression1}\ can be rewritten as
\LARGE
\begin{align}
    \sum_{i=1}^{\log_b(n)} a^{k-i}f(b^{i})
  &=   \sum_{i=0}^{\log_b(n)-1} a^{i}f(b^{k-i}) \\
  &=   \sum_{i=0}^{\log_b(n)-1} a^{i}f\left(\frac{b^{k}}{b^{i}}\right)\\
  &=   \sum_{i=0}^{\log_b(n)-1} a^{i}f\left(\frac{n}{b^{i}}\right)
\label{expression2}
\end{align}
\normalsize
Let's see what we can find out about this expression, using either
\ref{expression1}\ or \ref{expression2}, and how it compares to
\ref{base}. 

\pagebreak
\section{$f(n) = \Theta\left(n^{\log_b(a)}\right)$}
Let's first suppose
\LARGE
\begin{align}
  f(n) &= \Theta\left(n^{\log_b(a)}\right) \label{assumption1}
\end{align}
\normalsize
then
\LARGE
\begin{align}
f(b^i) &= \Theta\left((b^i)^{\log_b(a)}\right)\\
 &= \Theta\left(b^{i\log_b(a)}\right)\\
&= \Theta\left(\left(b^{\log_b(a)}\right)^i\right)\\
&= \Theta(a^i)
\end{align}
\normalsize
Plugging this back in our expression \ref{expression1}, gives
\LARGE
\begin{align}
\sum_{i=1}^{\log_b(n)} a^{k-i}f(b^{i}) &= \sum_{i=1}^{\log_b(n)} a^{k-i}\Theta(a^i)\\
&= \sum_{i=1}^{\log_b(n)} \Theta(a^k)\\
&= \sum_{i=1}^{\log_b(n)} \Theta(a^{\log_b(n)})\\
&= \sum_{i=1}^{\log_b(n)} \Theta(n^{\log_b(a)})\\
&= (\log_b(n)) \Theta(n^{\log_b(a)})\\
&=  \Theta(\lg(n)n^{\log_b(a)}) \label{fcomplexity}
\end{align}
\normalsize
Recall the expression for the complexity of $T(n)=T(b^k)$ in equation
\ref{tcomplexity}: 
\LARGE
\begin{align*}
T(b^k)
&=
n^{\log_b(a)}T(1) + \sum_{i=1}^{\log_b(n)} a^{k-i}f(b^{i})
\tag{\ref{tcomplexity}, revisited}
\end{align*}
\normalsize
Putting \ref{fcomplexity}\  into
\ref{tcomplexity}, the complexity of $T(n)$, 
shows that if
\LARGE
\begin{align*}
f(n) &= \Theta\left(n^{\log_b(a)}\right)\tag{\ref{assumption1}, revisited}
\end{align*}
\normalsize
then
\LARGE
\begin{align}
T(n) &= \Theta\left(\lg(n)n^{\log_b(a)}\right)
\end{align}

\pagebreak
\section{$f(n) = O\left(n^{\log_b(a)-\epsilon}\right)$}
\normalsize
Now let's suppose that $f(n)$ has some slightly lower complexity.
Suppose $0< \epsilon < \log_b(a)$ is some (small) positive real number, and
\LARGE
\begin{align}
  f(n) &= O\left(n^{\log_b(a)-\epsilon}\right)\label{assumption2}
\end{align}
\normalsize
then
\LARGE
\begin{align}
f\left(\frac{n}{b^i}\right) &=
O\left(
\left(
\frac{n}{b^i}
\right)^{\log_b(a) - \epsilon}
\right)
\\
&=
O\left(
\frac{n^{\log_b(a)-\epsilon}}{b^{i\log_b(a)-i\epsilon}}
\right)
\\
&=
O\left(
\left(
\frac{n^{\log_b(a)}} {n^{\epsilon}}
\right)
\left(
\frac{b^{i\epsilon}} {b^{i\log_b(a)}} 
\right)
\right)
\\
&=
O\left(
\frac{b^{\epsilon}} {b^{\log_b(a)}} 
\right)^i
\\
&=
O\left(
\frac{b^{\epsilon}} {a} 
\right)^i
\end{align}
\normalsize
Plugging this back into expression \ref{expression2} gives a geometric
series which can be solved:
\LARGE
\begin{align}
  \sum_{i=0}^{\log_b(n)-1} a^{i}f\left(\frac{n}{b^{i}}\right)
  &=
  O\left(
  \sum_{i=0}^{\log_b(n)-1} a^{i}
  \left(\frac{b^\epsilon}{a}\right)^i
  \right)
  \\
  &=
  O\left(
  \sum_{i=0}^{\log_b(n)-1} (b^\epsilon)^{i}
  \right)
  \\
  &=
  O  \left(
  \frac{(b^\epsilon)^{\log_b(n)} - 1}
       {b^\epsilon - 1}
       \right)
       \\
  &=
  O  \left(
  \frac{(n)^{\log_b(b^\epsilon)} - 1}
       {b^\epsilon - 1}
       \right)
       \\
  &=
  O  \left(
  \frac{n^\epsilon - 1}
       {b^\epsilon - 1}
       \right)\\
  &=
       O\left(n^\epsilon\right)\\
       &=
  O\left(n^{\log_b(a)}\right)       \label{fcomplexity2}
\end{align}
\normalsize
Since $b$ and $\epsilon$ are constants, and $0<\epsilon < \log_b(a)$.

Recall again the expression for the complexity of $T(n)=T(b^k)$ in equation
\ref{tcomplexity}: 
\LARGE
\begin{align*}
T(b^k)
&=
n^{\log_b(a)}T(1) + \sum_{i=1}^{\log_b(n)} a^{k-i}f(b^{i})
\tag{\ref{tcomplexity}, revisited}
\end{align*}
\normalsize
Plugging \ref{fcomplexity2} into \ref{tcomplexity}\ shows that if
\LARGE
\begin{align*}
f(n) = O\left(n^{\log_b(a)-\epsilon}\right)\tag{\ref{assumption2}, revisited}
  \end{align*}
\normalsize
then
\LARGE
\begin{align}
  T(n) = \Theta\left(n^{\log_b(a)}\right)
  \end{align}


\section{$f(n) = \Omega\left(n^{\log_b(a)+\epsilon}\right)$}
\normalsize
We leave it as an exercise to show that if $f$ has some slightly
higher complexity,
\LARGE
\begin{align}
  f(n) = \Omega\left(n^{\log_b(a)+\epsilon}\right)
\end{align}
\normalsize
and $af\left(\frac{n}{b}\right) \leq cf(n)$ for some $c$ and large
enough $n$, then
\LARGE
\begin{align}
  T(n) = \Theta(f(n))
\end{align}

\end{document}

