\documentclass{beamer}
\usepackage{cancel}
\usepackage{fancyvrb}
\usepackage{hyperref}
\usepackage{alltt}
\usepackage{graphicx}
\newcommand{\sect}[1]{
\section{#1}
\begin{frame}[fragile]\frametitle{#1}
}

\newtheorem{theo}{Theorem}[section]

\newcommand{\myfig}[1]{}%{\centerline{\includegraphics[scale=0.25]{figures/#1.png}}}
\newcommand{\myfigt}[1]{}%{\centerline{\includegraphics[scale=0.2]{figures/#1.png}}}
\newcommand{\myfigg}[1]{}%{\centerline{\includegraphics[scale=0.125]{figures/#1.png}}}

\newcommand{\trans}[5]{
\begin{tabular}{|c|c|c|c|c|}\hline
#1 & #2 & #3 & #4 & #5 \\\hline
\end{tabular}
}

\newcommand{\arr}{&\rightarrow&}
\newcommand{\darr}{&\Rightarrow&}
\newcommand{\ar}{\ensuremath{\rightarrow}}
\newcommand{\dar}{\ensuremath{\Rightarrow}}
\newcommand{\bee}{\begin{eqnarray*}}
\newcommand{\eee}{\end{eqnarray*}}
\newcommand{\lmb}{\ensuremath{\lambda}}

\newcommand{\bi}{\begin{itemize}}
\newcommand{\li}{\item}
\newcommand{\ei}{\end{itemize}}


\newcommand{\lrtable}{
\begin{tabular}{lrlr}
Stack & Input & Rule & Peek\\\hline
 $\$ aCd$ & $\$$&S\ar aCd & \\
 $\$ ac^+C$ & $d \$$ & C \ar cC & \\
 $\$ ac^+$&$d \$$ & C \ar c & d\\
 $\$ bCD$&$ \$$ & S\ar bCD & \\
 $\$ bCd$& $ \$$& D\ar d & \\
 $\$ bc^+C$&$d \$$ & C \ar cC & \\
 $\$ bc^+$&$d \$$& C \ar c& d\\\hline
\end{tabular}
}


\mode<presentation>
{
%  \usetheme{Madrid}
  % or ...

%  \setbeamercovered{transparent}
  % or whatever (possibly just delete it)
}

\usepackage[english]{babel}

\usepackage[latin1]{inputenc}

\title[Notes on Recursion Trees]
{
 Notes on Recursion Trees
}

\subtitle{} % (optional)

\author[Geoffrey Matthews]
{Geoffrey Matthews}
% - Use the \inst{?} command only if the authors have different
%   affiliation.

\institute[WWU/CS]
{
  Department of Computer Science\\
  Western Washington University
}
% - Use the \inst command only if there are several affiliations.
% - Keep it simple, no one is interested in your street address.

\date{\today}

% If you have a file called "university-logo-filename.xxx", where xxx
% is a graphic format that can be processed by latex or pdflatex,
% resp., then you can add a logo as follows:

%\pgfdeclareimage[height=0.5cm]{university-logo}{WWULogoProColor}
%\logo{\pgfuseimage{university-logo}}

% If you wish to uncover everything in a step-wise fashion, uncomment
% the following command: 

%\beamerdefaultoverlayspecification{<+->}

\begin{document}

\begin{frame}
  \titlepage
\end{frame}


\newcommand{\myref}[1]{\small\item\url{#1}}
\newcommand{\myreft}[1]{\footnotesize\item\url{#1}}

%\begin{frame}
%  \frametitle{Outline}
%  \tableofcontents
%  % You might wish to add the option [pausesections]
%\end{frame}

\sect{Recursion Tree for p. 89}
\begin{eqnarray*}
T(n) &=& 3T\left(\frac{n}{4}\right)+ cn^2\\\pause
T\left(\frac{n}{4}\right) &=& 3T\left(\frac{n}{4^2}\right)  + c\left(\frac{n}{4}\right)^2\\\pause
T(n) &=& 3\left[3T\left(\frac{n}{4^2}\right)  + c\left(\frac{n}{4}\right)^2\right]
+ cn^2\\\pause
&=& 3^2T\left(\frac{n}{4^2}\right) + cn^2\left(\frac{3}{4^2} + 1\right)\\\pause
T\left(\frac{n}{4^2}\right) &=&
3T\left(\frac{n}{4^3}\right) + c\left(\frac{n}{4^2}\right)^2\\\pause
T(n) &=& 3^2\left[3T\left(\frac{n}{4^3}\right) + c\left(\frac{n}{4^2}\right)^2\right] + cn^2\left(\frac{3}{4^2} + 1\right)\\\pause
&=& 3^3T\left(\frac{n}{4^3}\right) + cn^2\left(\frac{3^2}{(4^2)^2} + \frac{3}{4^2} + 1\right)\pause\mbox{\ \ \bf Aha!}\\\pause
&=& 3^3T\left(\frac{n}{4^3}\right) + cn^2\left(\frac{3^2}{(4^2)^2} + \frac{3^1}{(4^2)^1} + \frac{3^0}{(4^2)^0}\right)\\
\end{eqnarray*}
\end{frame}

\sect{Recursion Tree for p. 89}
\begin{eqnarray*}
T(n) &=& 3^3T\left(\frac{n}{4^3}\right) + cn^2\left(\frac{3^2}{(4^2)^2} + \frac{3^1}{(4^2)^1} + \frac{3^0}{(4^2)^0}\right)\\\pause
T(n) &=& 3^3T\left(\frac{n}{4^3}\right) + cn^2\sum_{i=0}^{2} \left(\frac{3}{16}\right)^i\\\pause
T(n) &=& 3^4T\left(\frac{n}{4^4}\right) + cn^2\sum_{i=0}^{3} \left(\frac{3}{16}\right)^i\\\pause
T(n) &=& 3^kT\left(\frac{n}{4^k}\right) + cn^2\sum_{i=0}^{k-1} \left(\frac{3}{16}\right)^i\\\pause
T(n) &=& 3^{\log_4 n}T\left(\frac{n}{4^{\log_4 n}}\right) + cn^2\sum_{i=0}^{\log_4 n -1} \left(\frac{3}{16}\right)^i\\\pause
T(n) &=& n^{\log_4 3}T(1) + cn^2\left(\frac{1-(3/16)^{\log_4 n}}{1-3/16}\right)\pause = \Theta(n^2)
\end{eqnarray*}
\end{frame}

\sect{{\bf Aha!} Revisited}
\begin{eqnarray*}
T(n) &=& 3T\left(\frac{n}{4}\right)+ cn^2\\
3T\left(\frac{n}{4}\right) &=& 3^2T\left(\frac{n}{4^2}\right)  + 3c\left(\frac{n}{4}\right)^2\\
3^2T\left(\frac{n}{4^2}\right) &=& 3^3T\left(\frac{n}{4^3}\right)  + 3^2c\left(\frac{n}{4^2}\right)^2\\
3^3T\left(\frac{n}{4^3}\right) &=& 3^4T\left(\frac{n}{4^4}\right)  + 3^3c\left(\frac{n}{4^3}\right)^2\\
&...&\\
3^{k-1}T\left(\frac{n}{4^{k-1}}\right) &=& 3^{k}T\left(\frac{n}{4^{k}}\right)  + 3^{k-1}c\left(\frac{n}{4^{k-1}}\right)^2\\
&...&\\
3^{\log_4 n-1}T\left(\frac{n}{4^{\log_4 n-1}}\right) &=& 3^{\log_4 n}T\left(\frac{n}{4^{\log_4 n}}\right)  + 3^{\log_4 n-1}c\left(\frac{n}{4^{\log_4 n-1}}\right)^2\\
\end{eqnarray*}
\end{frame}

\sect{{\bf Aha!} Revisited}
\begin{eqnarray*}
T(n) &=& \cancel{3T\left(\frac{n}{4}\right)}+ cn^2\\
\cancel{3T\left(\frac{n}{4}\right)} &=& \cancel{3^2T\left(\frac{n}{4^2}\right) } + 3c\left(\frac{n}{4}\right)^2\\
\cancel{3^2T\left(\frac{n}{4^2}\right)} &=& \cancel{3^3T\left(\frac{n}{4^3}\right)}  + 3^2c\left(\frac{n}{4^2}\right)^2\\
\cancel{3^3T\left(\frac{n}{4^3}\right)} &=& \cancel{3^4T\left(\frac{n}{4^4}\right)}  + 3^3c\left(\frac{n}{4^3}\right)^2\\
&...&\\
\cancel{3^{k-1}T\left(\frac{n}{4^{k-1}}\right)} &=& \cancel{3^{k}T\left(\frac{n}{4^{k}}\right)}  + 3^{k-1}c\left(\frac{n}{4^{k-1}}\right)^2\\
&...&\\
\cancel{3^{\log_4 n-1}T\left(\frac{n}{4^{\log_4 n-1}}\right)} &=& 3^{\log_4 n}T\left(\frac{n}{4^{\log_4 n}}\right)  + 3^{\log_4 n-1}c\left(\frac{n}{4^{\log_4 n-1}}\right)^2\pause\\
T(n) &=&3^{\log_4 n}T\left(\frac{n}{4^{\log_4 n}}\right)  + \sum_{i=0}^{\log_4 n-1}3^{i}c\left(\frac{n}{4^{i}}\right)^2\\
\end{eqnarray*}
\end{frame}

\end{document}
