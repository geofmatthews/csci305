\documentclass{beamer}
\usepackage{cancel}
\usepackage{fancyvrb}
\usepackage{graphicx}
\usepackage{cancel}
\usepackage{amsmath,amssymb,amsfonts}
\newcommand{\sect}[1]{
\section{#1}
\begin{frame}[fragile]\frametitle{#1}
}

\newtheorem{theo}{Theorem}[section]

\newcommand{\bi}{\begin{itemize}}
\newcommand{\ii}{\item}
\newcommand{\ei}{\end{itemize}}


\mode<presentation>
{
%  \usetheme{Madrid}
  % or ...

%  \setbeamercovered{transparent}
  % or whatever (possibly just delete it)
}

\usepackage[english]{babel}

\usepackage[latin1]{inputenc}

\title[Summations 2]
{
Summations 2
}

\subtitle{} % (optional)

\author
{Based on Notes by James Aspnes}
% - Use the \inst{?} command only if the authors have different
%   affiliation.

\institute[WWU/CS]
{
  Department of Computer Science\\
  Western Washington University
}
% - Use the \inst command only if there are several affiliations.
% - Keep it simple, no one is interested in your street address.

\date{\today}

% If you have a file called "university-logo-filename.xxx", where xxx
% is a graphic format that can be processed by latex or pdflatex,
% resp., then you can add a logo as follows:

%\pgfdeclareimage[height=0.5cm]{university-logo}{WWULogoProColor}
%\logo{\pgfuseimage{university-logo}}

% If you wish to uncover everything in a step-wise fashion, uncomment
% the following command: 

%\beamerdefaultoverlayspecification{<+->}

\begin{document}

\begin{frame}
  \titlepage
\end{frame}

\sect{Source}


These notes are based on chapter 6 of

\url{http://www.cs.yale.edu/homes/aspnes/classes/202/notes.pdf}

and

\url{http://www.cs.yale.edu/homes/aspnes/pinewiki/attachments/SummationNotation/summation-notation.pdf}


\end{frame}

\sect{Summations}

\[
\sum_{i=0}^{n}(2i+1) = 1 + 3 + 5 + 7 + ... + (2n+1)
\]
\end{frame}

\sect{Definition using recurrences}
\begin{align*}
  \sum_{i=a}^{b} f(i) &=
  \left\{
  \begin{array}{ll}
    0 & \mbox{if $b < a$} \\
    f(a) + \sum_{i=a+1}^{b} f(i) & \mbox{otherwise.}\\
  \end{array}
  \right.
\\\\
  \sum_{i=a}^{b} f(i) &=
  \left\{
  \begin{array}{ll}
    0 & \mbox{if $b < a$} \\
    \sum_{i=a}^{b-1} f(i) + f(b) & \mbox{otherwise.}\\
  \end{array}
  \right.
\end{align*}

\end{frame}

\sect{Scope}
\bi
\ii Scope extends to the first addition or subtraction symbol.
\ii Best to include parentheses to avoid confusion,
\ii or move trailing terms to the beginning.
\ei
\begin{align*}
  \sum_{i=1}^n i^2 + 1
  &= \left(  \sum_{i=1}^n i^2 \right) + 1 \\
  &= 1 + \left(  \sum_{i=1}^n i^2 \right) \\
  &= 1 +  \sum_{i=1}^n i^2  \\
  &\neq   \sum_{i=1}^n (i^2 + 1)
\end{align*}
\end{frame}

\sect{Summation is linear}
\begin{align*}
  \sum_{i=n}^{m} ax_i &= a  \sum_{i=n}^{m} x_i \\\\
  \sum_{i=n}^{m}( x_i + y_i) &= 
  \sum_{i=n}^{m} x_i  +
  \sum_{i=n}^{m} y_i \\
\end{align*}

\end{frame}

\sect{Multiple sums}
\bi
\ii The order is not important, provided the bounds of the inner sum
don't depend on the index of the outer sum:
\ei

\begin{align*}
  \sum_{i=a}^{b}\sum_{j=c}^{d} x_{ij} &=
  \sum_{j=c}^{d} \sum_{i=a}^{b} x_{ij}
\end{align*}
\end{frame}

\sect{Products of sums}

\begin{align*}
  (a + b)(x + y + z) &= ax + ay + az + bx + by + bz\\\\
  \left(\sum_{i=a}^{b}x_i\right)\left(\sum_{j=c}^{d} y_j\right) &=
  \sum_{i=a}^{b}\sum_{j=c}^{d} x_iy_j
\end{align*}
\end{frame}

\sect{Change of variables}
Let
\begin{align*}
  j &= i-1\\
  i &= j+1
\end{align*}
then,
\begin{align*}
  \sum_{i=1}^{n} (i-1) 
  &= \sum_{(j+1)=1}^{n} j \\
  &= \sum_{j=0}^{n-1} j \\
  &= \sum_{i=0}^{n-1} i \\
\end{align*}
\end{frame}

\sect{Change of variables}
Let
\begin{align*}
  j &= i+1\\
  i &= j-1
\end{align*}
then,
\begin{align*}
  \sum_{i=a}^{b} (i+1)^2 
  &= \sum_{(j-1)=a}^{b} j^2 \\
  &= \sum_{j=a+1}^{b+1} j^2 \\
  &= \sum_{i=a+1}^{b+1} i^2 \\
\end{align*}
\end{frame}
  
\sect{Sums over index sets}
\begin{align*}
  \sum_{i\in\{3,5,7\}} i^2 &= 3^2 + 5^2 + 7^2
  \\\\
  \sum_{A\subseteq S} |A|
  \\\\
  \sum_{\mbox{$p < 1000$, $p$ is prime}} p^2
\end{align*}
\end{frame}

\sect{Confusing index sets}
\[
\sum_{1 \leq i < j \leq n} \frac{i}{j}
\]
\bi
\ii The sum over all pairs of values $(i,j)$ such that
\bi
\ii $1\leq i$
\ii $i < j$
\ii $j \leq n$
\ei
with each pair appearing exactly once.
\ei
\end{frame}
    
\sect{Confusing index sets}
\[
\sum_{x\in A\subseteq S} |A|
\]
\bi
\ii The sum over all sets $A$ such that
\bi
\ii $x\in A$
\ii $A\subseteq S$
\ei
assuming that $x$ and $S$ are defined outside the summation.
\ei
\end{frame}

\sect{Sums without explicit bounds}

When the index set is understood from context some books use:
\[
\sum_i i^2
\]
Very sloppy, and discouraged.

\end{frame}

\sect{Double sums}
\[
\sum_{i=0}^{n}\sum_{j=0}^{i} (i+1)(j+1)
\]
When $n=1$ we get
\[
  (0+1)(0+1) + [ (1+1)(0+1) + (1+1)(1+1) ] = 7
\]
\end{frame}

\sect{Closed forms}
\begin{align*}
  \sum_{i=1}^{n} 1 &= n \\\\
  \sum_{i=1}^{n} i &= \frac{n(n+1)}{2}\\\\
  \sum_{i=0}^{n} r^i &= \frac{1-r^{n+1}}{1-r} \\
                   &= \frac{r^{n+1}-1}{r-1} \\
\end{align*}
\end{frame}

\sect{Quick ``proof''}
Start with
\[
\sum_{i=0}^{\infty} r^i = \frac{1}{1-r}
\]
Then
\begin{align*}
  \sum_{i=0}^n r^i
  &=   \sum_{i=0}^\infty r^i - r^{n+1} \sum_{i=0}^\infty r^i \\
  &= \frac{1}{1-r} - \frac{r^{n+1}}{1-r}\\
  &= \frac{1-r^{n+1}}{1-r}
\end{align*}
\end{frame}


\sect{Solving summations}
\begin{align*}
  \sum_{i=0}^n (3(2^n) + 5)
  &= 3  \sum_{i=0}^n 2^n + 5  \sum_{i=0}^n 1\\
  &= 3(2^{n+1} - 1) + 5(n+1) \\
  &= 3(2^{n+1}) + 5n + 2
\end{align*}
\end{frame}

\sect{Guess but verify}
\[
S(n) = \sum_{k=1}^n (2k-1)
\]
\[
\begin{array}{cr}
  n & S(n) \\\hline
  0 & 0 \\
  1 & 1 \\
  2 & 1 + 3 = 4\\
  3 & 4 + 5 = 9\\
  4 & 9 + 7 = 16\\
  5 & 16 + 9 = 25\\
\end{array}
\]

\begin{align*}
S(n) &= n^2  &\mbox{Guess!}
\end{align*}

\end{frame}
\sect{Prove guess by induction}
\[
S(n) = \sum_{k=1}^n (2k-1)
\stackrel{?}{\Longleftrightarrow}
S(n) = n^2 
\]
\begin{align*}
S(0) &=  \sum_{k=1}^0 (2k-1) = 0 & \mbox{Base case}\\
  S(n+1)
  &= \sum_{k=1}^{n+1} (2k-1) &\mbox{Step} \\
  &= \sum_{k=1}^{n} (2k-1) + (2(n+1) - 1)\\
  &= n^2 + 2(n+1) - 1 &\mbox{Inductive hypothesis}\\
  &= (n+1)^2
\end{align*}

\end{frame}

\sect{Strategies for asymptotic estimates}

Pull out constant factors

\begin{align*}
  \sum_{i=1}^n \frac{n}{i}
  &= n  \sum_{i=1}^n \frac{1}{i}\\
  &= nH_n\\
  &= \Theta(n\log n)
\end{align*}


\end{frame}

\sect{Strategies for asymptotic estimates}

Bound using geometric series

\begin{align*}
  \sum_{i=1}^n x^i &= \frac{x^{n+1}-1}{x-1} =\Theta(x^{n})
  \\\\
  \sum_{i=1}^n 2^i &= \Theta(2^{n})
  \\
  \sum_{i=1}^n 2^{-i} &= \Theta(1)
\end{align*}

\end{frame}
\sect{Strategies for asymptotic estimates}

Harmonic series

\begin{align*}
  \sum_{i=1}^n \frac{1}{i} &= H_n = \Theta(\lg n)
\end{align*}

\end{frame}
\sect{Strategies for asymptotic estimates}

Bound part of the sum.

\begin{align*}
  \sum_{i=1}^n i^3 &\leq   \sum_{i=1}^n n^3 \\
                  &= O(n^4)\\
  \sum_{i=1}^n i^3 &\geq   \sum_{i=n/2}^n i^3 \\
                  &\geq   \sum_{i=n/2}^n (n/2)^3 \\
                  &= \Omega(n^4)
\end{align*}

\end{frame}
\sect{Strategies for asymptotic estimates}

Integrate!

For increasing functions:

\begin{align*}
\int_{a-1}^b f(x)dx \leq \sum_{i=a}^b f(i) \leq \int_{a}^{b+1} f(x)dx
\end{align*}

 Most of the functions we see in algorithms yield to this method!
\end{frame}

\sect{Strategies for asymptotic estimates}

Grouping terms.

The standard trick for showing that the harmonic series is unbounded.

\begin{align*}
  &1 + 1/2 + (1/3 + 1/4) + (1/5 + 1/6 + 1/7 + 1/8) + ...\\
  &\geq\\
  &1 + 1/2 + (1/4 + 1/4) + (1/8 + 1/8 + 1/8 + 1/8) + ... \\
  &=\\
  &1 + 1/2 + 1/2 + 1/2 + ...
\end{align*}


\end{frame}

\sect{Final Notes}
In practice, almost any sum you come across will be of the form:
\[
\sum_{i=1}^n f(i)
\]
where either:
\begin{itemize}
  \item
    $f(n)$ is exponential.

    Then it's bounded by a geometric series and
the largest term dominates.
\item
  $f(n)$ is polynomial.

  Then $f(n/2) =\Theta(f(n))$ and the sum is
$\Theta(nf(n))$ using the lower bound:
\[
\sum_{i=n/2}^n f(n) =\Omega(nf(n))
\]
\end{itemize}
\end{frame}

\end{document}
