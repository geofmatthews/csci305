\documentclass{article}
\usepackage[margin=1in]{geometry}
\usepackage{multicol}
\usepackage{hyperref}
\usepackage{fancyvrb}
\usepackage{xcolor}

\newcommand{\nop}[1]{}

\begin{document}


\centerline{\Large \bf CSCI 305: Analysis of Algorithms I}

\centerline{\bf Syllabus,  Spring, 2018}

\begin{description}

\item[Instructor:] Geoffrey Matthews, x3797,
 geoffrey dot matthews at wwu dot edu

\item[Office hours:] MTWF 10:00-10:50, CF 469

\item[Lectures:] MTWF 12:00-12:50am, CF 227

\item [Webpages:] \mbox{}\begin{itemize}
\item\url{https://wwu.instructure.com} 
\item\url{https://github.com/geofmatthews/csci305}
\end{itemize}

\item
[Catalog Description:] Introduction to the analysis of algorithms and
data structures in a mathematically rigorous fashion. Mathematical
fundamentals, counting, discrete probability, asymptotic notation,
recurrences, loop invariants. Worst-case, probabilistic and amortized
analysis techniques applied to sorting algorithms and classic data
structures such as heaps, trees and hash tables. Design techniques
such as branch and bound, divide and conquer, will be introduced as
will correctness proofs for algorithms.

\item[Course Outcomes:] On completion of this course students will
  demonstrate:
  \begin{itemize}
\item
Basic understanding of the mathematical concepts of asymptotic
notation, recurrence relations 
and loop invariants.
\item
 Basic understanding of worst-case, probabilistic and amortized
analysis techniques.
\item
 Thorough understanding of the complexity of sorting algorithms and
common data structures, 
such as heaps, trees and hash tables.
\item
   Thorough understanding of divide and conquer.
  \item
 Basic understanding of the formulation and analysis of probabilistic
algorithms.
\item
 The ability to analyze and formulate solutions for abstract
problems.
\item
 The ability to derive the time and space complexity of basic
algorithms.
\item
 The ability to use mathematical reasoning in correctness proofs of
algorithms.
  \end{itemize}
  

  \item[Text:] {\em Introduction to Algorithms,} 3rd ed., Cormen,
    Leiserson, Rivest, and Stein.


\item[Quizzes:]  Pop quizzes may be handed out at any time.  After
  working on them individually we will solve them together in class.
  They must be turned in during class for credit.  No makeups.

\item[Exams:] A midterm and a final, according to the schedule below.
  The final will be cumulative.

  Exams are closed book, with the exception that you may consult two
  pieces of paper during the exam.  You may print or write whatever
  you wish on these pages.

  If you have to miss the midterm exam, notify me as soon as
  possible.  I will handle each case individually.

\item[Homework:] Homework must be typeset using \LaTeX.  The easiest
  way to get started is using one of the online \LaTeX\ tools, such as
  \url{https://www.overleaf.com/}.

   Homework must be submitted to canvas by midnight of the due date.
   Late homework will not be accepted.  There will always be
   sufficient time given to complete the homework in a timely manner
   if you don't wait to the last night.

   Homework may not be graded promptly, but solutions will be posted
   online shortly after the due date.  Only a subset of the submitted
   problems will be graded.

   There may be an assignment due during dead week.

 \item[Assessment:] Assessment will be based on homework, quizzes, the
   midterm, and the final.  There are no extra credit opportunities in
   this class.  Assignment of $\pm$ grades is at the discretion of the
   instructor. 

   \begin{tabular}{|c|c|c|c|}\hline
     Homework & Quizzes & Midterm & Final\\\hline
       35\% & 5\% & 20\%& 40\%\\\hline
   \end{tabular}
   \hfill
   \begin{tabular}{|c|c|c|c|c|}\hline
     A & B & C & D & F\\\hline
    100-90& 89-80 & 79-70 & 69-60 & 59-0\\\hline
   \end{tabular}


\item [Schedule:]\mbox{}

\begin{tabular}{r|lll}                      
     Week & Text section & Notes & Important dates \\\hline  
     April 1--7&  1, 2 & 000--020 \\
     April 8--14& A  & 030--040\\
     April 15--21 & 3 & 050--070 \\
     April 22--28 & 4 & 080--110 \\
     April 29--May 4 & 5, C & 120--130 & \\
     May 6--12 &&& Midterm, Wednesday, May 9\\
     May 13--19 & 6, 7 & 140--150 \\ 
     May 20--26 &  8 (\S 1,2,3), 11 (\S 1,2) & 160--170 \\
     May 27--June 2 & 12 (\S 1,2,3), 13& 180--190 & Monday, May 28, Holiday\\
     June 3--9 &17 (\S 1,2,3)& 200 \\
     June 10--17 &&&  Final Exam, Tuesday, June 12, 8:00am \\
\end{tabular}


\item [Academic dishonesty:] Academic dishonesty policy and
  procedure is discussed in the University Catalog, Appendix D.  All
  students should read this section of the catalog.  Academic
  dishonesty consists of misrepresentation by deception or other
  fraudulent means.  In computer science courses this frequently takes
  the form of copying another's program, either a fellow student's
  program, or copying one from the web.  Due diligence should be
  exercised in the labs at all times, since both copying and letting
  someone else copy your program are equally culpable.  Do not walk
  away from your computer in the lab without logging out or locking
  the screen.  Do not print out code and then throw it away in the lab
  trash cans. Do not share files, even if it is just to ``show them
  something.''  Describe it in words, or talk to them in person, never
  share code.

\item [Collaboration:] Collaboration with your fellow students is
  a good way to learn.  Feel free to share ideas, solve problems, and
  discuss your programs with other students.  However, collaboration
  is {\em not} copying.  All code should be original.  Remember the
  \fbox{\bf Long Term Memory Rule}: After discussing homework with
  another student, each of you must destroy all written notes,
  pictures, files that you shared, erase the board, {\em
    etc.}.  After that, you must watch a rerun of {\em the Simpson's},
  play a round of ping-pong, go for a walk, or do something else
  unrelated, for half an hour.  Then you can take the knowledge you
  gained from another student and put it to work, since it is now not
  copying, but learning.  You have made it your own.

\end{description}

\end{document}
